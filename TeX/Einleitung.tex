\chapter{Einleitung}

\lettrine{B}{itcoin} hat sich innerhalb des vergangenen Jahrzehnts von einem experimentellen digitalen Zahlungssystem zu einem weltweit gehandelten Finanzmarktinstrument mit makroökonomischer Relevanz entwickelt \citep{Baur2018}.
Während Kryptowährungen in ihrer Anfangsphase vor allem als technische Innovation betrachtet wurden, lässt sich ungefähr ab dem Jahr 2015 ein Wandel erkennen, bei dem Kryptowährungen verstärkt als eigene Anlageklasse etabliert haben \citep{Corbet2019}.
Dies zeigt sich unter anderem in einem deutlich steigenden Handelsvolumen sowie einer wachsenden Marktkapitalisierung.

Bitcoin wurde ursprünglich als vollständig dezentrales elektronisches Zahlungssystem entworfen.
In seinem grundlegenden Whitepaper beschreibt \textcite{Nakamoto2008} Bitcoin als \textit{\enquote{a peer-to-peer electronic cash system}}, das es ermöglicht, Online-Zahlungen direkt zwischen zwei Beteiligten durchzuführen, ohne dass eine vertrauenswürdige dritte Instanz erforderlich ist.
Die Idee, Zahlungen ohne Banken oder andere Vermittler abzuwickeln, stellte dabei ein zentrales Merkmal von Bitcoin dar.
Im Laufe der Zeit entwickelte sich Bitcoin jedoch zunehmend von einem Zahlungsmittel hin zu einem digitalen Anlageobjekt, welches heute vor allem als Wertaufbewahrungsmittel betrachtet wird \citep{Yermack2015}.

Parallel zu dieser Entwicklung ist eine zunehmende Akzeptanz von Bitcoin zu beobachten.
Neben privaten Anlegern investieren auch immer mehr Unternehmen und professionelle Marktteilnehmer in Bitcoin.
Ein weiterer wichtiger Schritt war die Zulassung von Bitcoin-ETFs, die den Zugang zu Bitcoin insbesondere für private Anleger erleichtert und die Einbindung in bestehende Finanzmärkte weiter vorangetrieben haben \citep{Corbet2019}.

Vor diesem Hintergrund untersucht diese Seminarbeit den Zusammenhang zwischen ausgewählten makroökonomischen Variablen und der Preisentwicklung von Bitcoin.
Ziel ist es, zu analysieren, inwiefern Veränderungen zentraler wirtschaftlicher Rahmenbedingungen mit Änderungen des Bitcoin Preises in Verbindung stehen.
Die empirische Analyse orientiert sich an der Methodik von \textcite{Baur2018}, in dessem Paper Bitcoin sowohl als Zahlungsmittel als auch als spekulatives Anlageobjekt betrachtet wird.
