\chapter{Empirische Analyse}

Ziel der empirischen Analyse ist es, den Zusammenhang zwischen ausgewählten makroökonomischen Variablen und der Preisentwicklung von Bitcoin zu untersuchen. Aufbauend auf den im vorherigen Kapitel beschriebenen Daten und der Methodik erfolgt die empirische Auswertung schrittweise.

Als erstes werden die betrachteten Zeitreihen mithilfe deskriptiver Statistik und grafischer Darstellungen analysiert. Anschließend werden die zeitlichen Eigenschaften der Daten untersucht, bevor mithilfe von Stationaritätstests die Voraussetzungen für die nachfolgende Regressionsanalyse geprüft werden. Abschließend werden Regressionsmodelle geschätzt, um den Einfluss makroökonomischer Faktoren auf Kryptowährungsrenditen zu quantifizieren. Der Fokus liegt dabei auf Bitcoin, während Ethereum ergänzend als Robustheitstest herangezogen wird.

\section{Deskriptive Statistik}
Hier werden die aufbereiteten Daten deskriptiv ausgewertet um einen Überblick über die grundlegenden Eigenschaften zu gewinnen und bezieht sich auf den gesamten Beobachtungszeitraum von Januar~2015 bis Dezember~2025. Die Daten basieren auf den monatlichen Renditen von Bitcoin, Ethereum und S\&P~500-Renditen sowie ausgewählten makroökonomischen Variablen wie Inflation, Zinsen, Geldmengenwachstum, US-Dollar-Index.

\begin{table}[ht]
    \centering
    \caption{Deskriptive Statistik der verwendeten Variablen}
    \label{tab:deskriptiv}
    \begin{tabular}{lrrrrrr}
\toprule
 & Mean & Std. & Min & Max & Skewness & Kurtosis \\
Variable &  &  &  &  &  &  \\
\midrule
btc\_logret & 0.05 & 0.17 & -0.37 & 0.67 & 0.45 & 1.00 \\
eth\_logret & 0.03 & 0.23 & -0.46 & 0.66 & 0.18 & 0.01 \\
sp500\_logret & 0.01 & 0.04 & -0.21 & 0.08 & -2.19 & 10.60 \\
usdindex\_logret & 0.00 & 0.01 & -0.02 & 0.04 & 0.17 & -0.42 \\
inflation & 0.00 & 0.00 & -0.01 & 0.01 & 0.41 & 3.38 \\
m2\_growth & 0.00 & 0.01 & -0.01 & 0.06 & 4.12 & 24.88 \\
fedfunds & 1.97 & 1.92 & 0.05 & 5.33 & 0.65 & -1.11 \\
d\_fedfunds & 0.03 & 0.18 & -0.93 & 0.70 & -0.48 & 9.71 \\
\bottomrule
\end{tabular}

\end{table}

Die Werte aus der Tabelle \ref{tab:deskriptiv} zeigen, eine höhere Volatilität von Kryptowährungen im Vergleich zu traditionellen Anlageklassen wie dem S\&P~500. Bitcoin und Ethereum weisen deutlich höhere Streuungen und Extremwerte auf, während US-Dollar, Inflation, Geldmengenwachstum und Zinssätze im Vergleich stabiler erscheinen.

\newpage
\section{Zeitreihenanalyse}
Die Zeitreihenanalyse untersucht die zeitliche Entwicklung verschiedener Variablen. Grafische Darstellungen helfen z.B. um Trends, strukturelle Änderungen oder auffälige Marktphasen zu finden.

\begin{figure}[htbp]
    \centering
    \includegraphics[width=\linewidth]{../src/figures/btc_vs_sp500_returns.pdf}
    \caption{Bitcoin vs. S\&P~500 Renditen (2015-2025)}
    \label{fig:btc_vs_sp500_returns}
\end{figure}

Der Graph von \ref{fig:btc_vs_sp500_returns} zeigt die monatlichen logarithmischen Renditen von Bitcoin und dem S\&P~500 im Vergleich. Die Renditen von Bitcoin zeigen deutlich stärkere Schwankungen als die vom S\&P~500. Außerdem gibt es bei Bitcoin besonders hohe positive sowie negative Ausschläge. Im Gegensatz dazu verzeichnet der S\&P~500 überwiegend positive Renditen mit geringerer Volatilität. Dies deutet auf eine insgesamt hohe Volatilität von Bitcoin hin. Ein langfristiger Trend lässt sich in den dargestellten Renditen jedoch nicht erkennen.

\newpage

\begin{figure}[htbp]
    \centering
    \includegraphics[width=\linewidth]{../src/figures/fedfunds_inflation.pdf}
    \caption{Zinsen und Inflation (2015-2025)}
    \label{fig:fedfunds_inflation}
\end{figure}

Der Graph von \ref{fig:fedfunds_inflation} stellt die Entwicklung der US-amerikanischen Zinssätze sowie der Inflationsrate im Zeitraum von 2015 bis 2025 dar. Die Zinssätze verbleiben bis etwa 2021 auf einem sehr niedrigen Niveau, bevor ab dem Jahr 2022 ein deutlicher und schneller Anstieg zu beobachten ist, der in den Folgejahren auf einem hohen Niveau bleibt. Die Inflationsrate zeigt im Vergleich dazu stärkere kurzfristige Schwankungen, mit einem starken Rückgang im Jahr 2020 sowie einem ausgeprägten Anstieg in den Jahren 2021 und 2022. In den darauffolgenden Jahren ist ein Rückgang der Inflationsrate erkennbar.

\section{Stationaritätstests}
Zur Überprüfung der statistischen Eigenschaften der Zeitreihen werden Augmented-Dickey-Fuller-Tests (ADF-Tests) durchgeführt. Ziel ist es zu prüfen, ob die betrachteten Variablen stationär sind und somit die Voraussetzungen für die nachfolgende Regressionsanalyse erfüllen.

\begin{table}[ht]
    \centering
    \caption{Ergebnisse der ADF-Tests}
    \label{tab:adf_results}
    \begin{tabular}{lrl}
\toprule
Variable & p-Wert & Stationär (5\%) \\
\midrule
btc\_logret & 0.0000 & Ja \\
sp500\_logret & 0.0000 & Ja \\
usdindex\_logret & 0.0000 & Ja \\
inflation & 0.2915 & Nein \\
m2\_growth & 0.0013 & Ja \\
fedfunds & 0.4394 & Nein \\
\bottomrule
\end{tabular}

\end{table}

Renditereihen wie von Bitcoin oder S\&P~500 sind in der Regel stationär, da sie um ein konstantes Niveau schwanken. Inflation oder Zinsen sind häufig nicht stationär da sie abhängig von globalen Ereignissen sind \citep{Enders2014}.

\section{Regressionsanalyse}
Zur Untersuchung des Einflusses makroökonomischer Variablen auf die Renditen von Bitcoin wird ein lineares Regressionsmodell geschätzt. Die abhängige Variable ist die monatliche Rendite von Bitcoin, während ausgewählte makroökonomische Indikatoren als erklärende Variablen dienen.
Das Regressionsmodell hat folgende Form:
\[
    r^{BTC}_t = \alpha + \beta_1 r^{SP500}_t + \beta_2 r^{USD}_t + \beta_3 \pi_t + \beta_4 \Delta M2_t + \beta_5 i_t + \varepsilon_t
\]

\begin{table}[ht]
    \centering
    \caption{Regressionsergebnisse}
    \label{tab:regression}
    \begin{tabular}{lrrrl}
\toprule
Variable & Koeffizient & Std.-Fehler & p-Wert & Signifikanz \\
\midrule
const & 0.0540 & 0.0245 & 0.0290 & ** \\
sp500\_logret & 2.2672 & 0.4367 & 0.0000 & *** \\
usdindex\_logret & 1.0675 & 1.2713 & 0.4027 &  \\
inflation & -8.1552 & 5.5651 & 0.1454 &  \\
m2\_growth & -1.0181 & 2.1137 & 0.6309 &  \\
d\_fedfunds & -0.0832 & 0.0963 & 0.3890 &  \\
\bottomrule
\end{tabular}

\end{table}

Die Ergebnisse der Regressionsanalyse sind in Tabelle~\ref{tab:regression} dargestellt.

Ein deutlich positiver und hochsignifikanter Zusammenhang zeigt sich zwischen den Renditen des S\&P~500 und den Bitcoin-Renditen. Bitcoin hat im betrachteten Zeitraum eher den Charakter einer Risikoanlage und entwickelt sich insbesondere in Phasen steigender Aktienmärkte positiv. Dieses Ergebnis steht im Einklang mit der bestehenden Literatur, die Bitcoin überwiegend als spekulatives Finanzmarktinstrument und weniger als sicheren Hafen einordnet \cite{Baur2018,Corbet2018}.

Für den US-Dollar-Index ergibt sich hingegen kein statistisch signifikanter Zusammenhang mit den Bitcoin Renditen. Ebenso weisen weder das Geldmengenwachstum noch das Zinsniveau einen signifikanten Einfluss auf die Renditen von Bitcoin auf. Diese Erkenntnisse deuten darauf hin, dass die Größen im untersuchten Zeitraum nur eine untergeordnete Rolle für die kurzfristige Renditeentwicklung von Bitcoin gespielt haben.

Die Inflationsrate zeigt einen negativen Zusammenhang mit den Bitcoin-Renditen, der jedoch lediglich auf dem 10\%-Signifikanzniveau schwach signifikant ist. Damit liefern die Ergebnisse keine eindeutige Evidenz für eine Funktion von Bitcoin als Inflationsschutz. Dieses Ergebnis stimmt mit der bestehenden Literatur überein, die Bitcoin keine stabilen geldähnlichen Eigenschaften zuschreiben und seine Eignung als Absicherungsinstrument gegen Inflation infrage stellen \cite{Baur2018,Yermack2015}.
\newpage
\subsubsection*{Robustheitsanalyse mit Ethereum}
Zur Überprüfung der Robustheit der Ergebnisse wird die Regressionsanalyse zusätzlich für Ethereum durchgeführt.
\begin{table}[ht]
    \centering
    \caption{Regressionsergebnisse für Ethereum}
    \label{tab:regression_eth}
    \begin{tabular}{lrrrl}
\toprule
Variable & Koeffizient & Std.-Fehler & p-Wert & Signifikanz \\
\midrule
const & -0.0092 & 0.0350 & 0.7933 &  \\
sp500\_logret & 3.1415 & 0.5782 & 0.0000 & *** \\
usdindex\_logret & -0.4603 & 1.8764 & 0.8068 &  \\
inflation & -0.4183 & 7.5428 & 0.9559 &  \\
m2\_growth & 1.5019 & 2.6480 & 0.5720 &  \\
d\_fedfunds & -0.0617 & 0.1222 & 0.6151 &  \\
\bottomrule
\end{tabular}

\end{table}

Die Ergebnisse bestätigen die Bitcoin Analyse. Der S\&P~500 hat einen signifikanten und positiven Einfluss auf die Renditen von Ethereum auf. Der geschätzte Koeffizient fällt dabei sogar höher aus als im Bitcoin Modell, was auf eine noch stärkere Einbindung von Ethereum in das allgemeine Marktgeschehen hindeutet.
Inflation, Geldmengenwachstum, US-Dollar-Index und Zinssätze, zeigen keinen statistisch signifikanten Einfluss auf die Ethereum Renditen. Damit bleiben sowohl Richtung als auch Signifikanz der Koeffizienten über beide Kryptowährungen hinweg konsistent.
