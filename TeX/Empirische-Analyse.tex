\chapter{Empirische Analyse}

Ziel der empirischen Analyse ist es, den Zusammenhang zwischen ausgewählten makroökonomischen Variablen und der Preisentwicklung von Bitcoin zu untersuchen. Aufbauend auf den im vorherigen Kapitel beschriebenen Daten und der Methodik erfolgt die empirische Auswertung schrittweise.

Als erstes werden die betrachteten Zeitreihen mithilfe deskriptiver Statistik und grafischer Darstellungen analysiert. Anschließend werden die zeitlichen Eigenschaften der Daten untersucht, bevor mithilfe von Stationaritätstests die Voraussetzungen für die nachfolgende Regressionsanalyse geprüft werden. Abschließend werden Regressionsmodelle geschätzt, um den Einfluss makroökonomischer Faktoren auf Kryptowährungsrenditen zu quantifizieren.

Um mögliche strukturelle Veränderungen im wirtschaftlichen und geldpolitischen Umfeld zu berücksichtigen, erfolgt die Analyse sowohl für den Gesamtzeitraum als auch getrennt nach vier Teilperioden: (i) eine Vorkrisenphase von Januar~2015 bis Dezember~2019, (ii) die Phase des pandemiebedingten Markteinbruchs im ersten Halbjahr~2020, (iii) eine Phase expansiver Geldpolitik von Juli~2020 bis Dezember~2021 sowie (iv) eine Phase steigender Inflation und geldpolitischer Straffung von Januar~2022 bis Dezember~2025. Dadurch kann untersucht werden, ob sich die Zusammenhänge zwischen makroökonomischen Variablen und Kryptowährungen in unterschiedlichen Marktphasen unterscheiden. Der Fokus liegt dabei auf Bitcoin, während Ethereum ergänzend als Robustheitstest herangezogen wird.

\section{Deskriptive Statistik}
Hier werden die aufbereiteten Daten deskriptiv ausgewertet um einen Überblick über die grundlegenden Eigenschaften zu gewinnen und bezieht sich auf den gesamten Beobachtungszeitraum von Januar~2015 bis Dezember~2025. Die Daten basieren auf den monatlichen Renditen von Bitcoin, Ethereum und S\&P~500-Renditen sowie ausgewählten makroökonomischen Variablen wie Inflation, Zinsen, Geldmengenwachstum, US-Dollar-Index.

\begin{table}[ht]
    \centering
    \caption{Deskriptive Statistik der verwendeten Variablen}
    \label{tab:deskriptiv}
    \begin{tabular}{lrrrrrr}
\toprule
 & Mean & Std. & Min & Max & Skewness & Kurtosis \\
Variable &  &  &  &  &  &  \\
\midrule
btc\_logret & 0.05 & 0.17 & -0.37 & 0.67 & 0.45 & 1.00 \\
eth\_logret & 0.03 & 0.23 & -0.46 & 0.66 & 0.18 & 0.01 \\
sp500\_logret & 0.01 & 0.04 & -0.21 & 0.08 & -2.19 & 10.60 \\
usdindex\_logret & 0.00 & 0.01 & -0.02 & 0.04 & 0.17 & -0.42 \\
inflation & 0.00 & 0.00 & -0.01 & 0.01 & 0.41 & 3.38 \\
m2\_growth & 0.00 & 0.01 & -0.01 & 0.06 & 4.12 & 24.88 \\
fedfunds & 1.97 & 1.92 & 0.05 & 5.33 & 0.65 & -1.11 \\
d\_fedfunds & 0.03 & 0.18 & -0.93 & 0.70 & -0.48 & 9.71 \\
\bottomrule
\end{tabular}

\end{table}

Die Werte aus der Tabelle \ref{tab:deskriptiv} zeigen, eine höhere Volatilität von Kryptowährungen im Vergleich zu traditionellen Anlageklassen wie dem S\&P~500. Bitcoin und Ethereum weisen deutlich höhere Streuungen und Extremwerte auf, während US-Dollar, Inflation, Geldmengenwachstum und Zinssätze im Vergleich stabiler erscheinen.


\section{Zeitreihenanalyse}
Die Zeitreihenanalyse untersucht die zeitliche Entwicklung verschiedener Variablen. Grafische Darstellungen helfen z.B. um Trends, strukturelle Änderungen oder auffälige Marktphasen zu finden.

\begin{figure}[htbp]
    \centering
    \includegraphics[width=\linewidth]{../src/figures/btc_vs_sp500_returns.pdf}
    \caption{Bitcoin vs. S\&P~500 Renditen (2015-2025)}
    \label{fig:btc_vs_sp500_returns}
\end{figure}

Der Graph von \ref{fig:btc_vs_sp500_returns} zeigt die monatlichen logarithmischen Renditen von Bitcoin und dem S\&P~500 im Vergleich. Die Renditen von Bitcoin zeigen deutlich stärkere Schwankungen als die vom S\&P~500. Außerdem gibt es bei Bitcoin besonders hohe positive sowie negative Ausschläge. Im Gegensatz dazu verzeichnet der S\&P~500 überwiegend positive Renditen mit geringerer Volatilität. Dies deutet auf eine insgesamt hohe Volatilität von Bitcoin hin. Ein langfristiger Trend lässt sich in den dargestellten Renditen jedoch nicht erkennen.

\begin{figure}[htbp]
    \centering
    \includegraphics[width=\linewidth]{../src/figures/fedfunds_inflation.pdf}
    \caption{Zinsen und Inflation (2015-2025)}
    \label{fig:fedfunds_inflation}
\end{figure}

Der Graph von \ref{fig:fedfunds_inflation} stellt die Entwicklung der US-amerikanischen Zinssätze sowie der Inflationsrate im Zeitraum von 2015 bis 2025 dar. Die Zinssätze verbleiben bis etwa 2021 auf einem sehr niedrigen Niveau, bevor ab dem Jahr 2022 ein deutlicher und schneller Anstieg zu beobachten ist, der in den Folgejahren auf einem hohen Niveau bleibt. Die Inflationsrate zeigt im Vergleich dazu stärkere kurzfristige Schwankungen, mit einem starken Rückgang im Jahr 2020 sowie einem ausgeprägten Anstieg in den Jahren 2021 und 2022. In den darauffolgenden Jahren ist ein Rückgang der Inflationsrate erkennbar.

\section{Stationaritätstests}
\section{Regressionsanalyse}