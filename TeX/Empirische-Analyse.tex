\chapter{Empirische Analyse}

Ziel der empirischen Analyse ist es, den Zusammenhang zwischen ausgewählten makroökonomischen Variablen und der Preisentwicklung von Bitcoin zu untersuchen. Aufbauend auf den im vorherigen Kapitel beschriebenen Daten und der Methodik erfolgt die empirische Auswertung schrittweise.

Als erstes werden die betrachteten Zeitreihen mithilfe deskriptiver Statistik und grafischer Darstellungen analysiert. Anschließend werden die zeitlichen Eigenschaften der Daten untersucht, bevor mithilfe von Stationaritätstests die Voraussetzungen für die nachfolgende Regressionsanalyse geprüft werden. Abschließend werden Regressionsmodelle geschätzt, um den Einfluss makroökonomischer Faktoren auf Kryptowährungsrenditen zu quantifizieren.

Um mögliche strukturelle Veränderungen im wirtschaftlichen und geldpolitischen Umfeld zu berücksichtigen, erfolgt die Analyse sowohl für den Gesamtzeitraum als auch getrennt nach vier Teilperioden: (i) eine Vorkrisenphase von Januar~2015 bis Dezember~2019, (ii) die Phase des pandemiebedingten Markteinbruchs im ersten Halbjahr~2020, (iii) eine Phase expansiver Geldpolitik von Juli~2020 bis Dezember~2021 sowie (iv) eine Phase steigender Inflation und geldpolitischer Straffung von Januar~2022 bis Dezember~2025. Dadurch kann untersucht werden, ob sich die Zusammenhänge zwischen makroökonomischen Variablen und Kryptowährungen in unterschiedlichen Marktphasen unterscheiden. Der Fokus liegt dabei auf Bitcoin, während Ethereum ergänzend als Robustheitstest herangezogen wird.

\section{Deskriptive Statistik}
Hier werden die aufbereiteten Daten deskriptiv ausgewertet um einen Überblick über die grundlegenden Eigenschaften zu gewinnen und bezieht sich auf den gesamten Beobachtungszeitraum von Januar~2015 bis Dezember~2025. Die Daten basieren auf den monatlichen Renditen von Bitcoin, Ethereum und S\&P~500-Renditen sowie ausgewählten makroökonomischen Variablen wie Inflation, Zinsen, Geldmengenwachstum, US-Dollar-Index.

\begin{table}[ht]
    \centering
    \caption{Deskriptive Statistik der verwendeten Variablen}
    \label{tab:deskriptiv}
    \begin{tabular}{lrrrrrr}
\toprule
 & Mean & Std. & Min & Max & Skewness & Kurtosis \\
Variable &  &  &  &  &  &  \\
\midrule
btc\_logret & 0.05 & 0.17 & -0.37 & 0.67 & 0.45 & 1.00 \\
eth\_logret & 0.03 & 0.23 & -0.46 & 0.66 & 0.18 & 0.01 \\
sp500\_logret & 0.01 & 0.04 & -0.21 & 0.08 & -2.19 & 10.60 \\
usdindex\_logret & 0.00 & 0.01 & -0.02 & 0.04 & 0.17 & -0.42 \\
inflation & 0.00 & 0.00 & -0.01 & 0.01 & 0.41 & 3.38 \\
m2\_growth & 0.00 & 0.01 & -0.01 & 0.06 & 4.12 & 24.88 \\
fedfunds & 1.97 & 1.92 & 0.05 & 5.33 & 0.65 & -1.11 \\
d\_fedfunds & 0.03 & 0.18 & -0.93 & 0.70 & -0.48 & 9.71 \\
\bottomrule
\end{tabular}

\end{table}

\section{Zeitreihenanalyse}
\section{Stationaritätstests}
\section{Regressionsanalyse}