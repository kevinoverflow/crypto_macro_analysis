\chapter{Theoretischer Hintergrund}
In diesem Kapitel werden die theoretischen Grundlagen für diese Seminararbeit und die bisherige Forschung zu Kryptowährungen und deren Beziehung zu makroökonomischen Variablen vorgestellt.

\section{Kryptowährungen als Finanzmarktsegment}

Kryptowährungen sind digitale, kryptographisch abgesicherte Währungen, die auf dezentralen Netzwerken basieren und ohne eine zentrale Autorität funktionieren \citep{Nakamoto2008}. Seit der Einführung von Bitcoin im Jahr 2009 hat sich der Markt für Kryptowährungen rasant entwickelt, sodass heute mehrere tausend unterschiedliche Coins und Tokens existieren, die auf vielfältige Weise genutzt werden. Während Kryptowährungen in ihrer frühen Phase primär als technologische Innovation wahrgenommen wurden, haben sie sich im Zeitverlauf zunehmend zu einem eigenständigen Segment der Finanzmärkte entwickelt.

Neben ihrer ursprünglichen Funktion als alternatives Zahlungsmittel sind Kryptowährungen mittlerweile fester Bestandteil digitaler und finanzieller Infrastrukturen. Die zugrunde liegende Blockchain-Technologie bildet die Basis für neue Formen der Wertübertragung, dezentrale Finanzanwendungen sowie die Integration von Kryptowährungen in bestehende Zahlungssysteme. Damit sind Kryptowährungen nicht mehr ausschließlich als isolierte digitale Währungen zu betrachten, sondern als Teil eines breiteren technologischen und ökonomischen Ökosystems.

Die heutige Relevanz von Kryptowährungen liegt dabei vor allem im finanziellen Bereich. Kryptowährungen werden aktiv an globalen Märkten gehandelt, weisen eine hohe Volatilität auf und unterscheiden sich durch ihren vergleichsweise geringen Regulierungsgrad von klassischen Finanzmarktinstrumenten \citep{Baur2018}. Dieser geringe Regulierungsgrad bedeutet zugleich, dass Kryptowährungen in bestimmten Kontexten zur Umgehung staatlicher Kontrolle genutzt werden können, etwa im Schwarzmarkt- oder Darknet-Handel, wo digitale Vermögenswerte als alternatives Zahlungsmittel eingesetzt werden \citep{EuropolCryptoCrime}.

Darüber hinaus ist Bitcoin inzwischen Bestandteil der Finanzstrategien zahlreicher Unternehmen und Finanzintermediäre. So halten börsennotierte Unternehmen wie MicroStrategy und Tesla Bitcoin als Teil ihrer Unternehmensreserven \citep{BitcoinTreasuries2024}. Neben Unternehmen mit direkten Bitcoin-Beständen haben Zahlungsdienstleister wie PayPal sowie Neobanken wie Revolut Kryptowährungen in ihre Finanzdienstleistungen integriert, beispielsweise durch Wallet-Funktionen oder Brokerage-Dienste \citep{PayPalCrypto,RevolutCrypto}. Ergänzt wird diese Entwicklung durch die Einführung regulierter Finanzprodukte wie Bitcoin-ETFs. Im Oktober 2021 wurde mit dem ProShares Bitcoin Strategy ETF der erste Bitcoin-ETF aufgelegt, der auf Bitcoin-Futures basiert \citep{ProSharesETF2021}. Darüber hinaus genehmigte die US-amerikanische Börsenaufsichtsbehörde SEC Anfang 2024 mehrere Spot-Bitcoin-ETFs, welche den tatsächlichen Bitcoin-Preis abbilden und institutionellen Investoren einen regulierten Zugang zu Bitcoin ermöglichen \citep{SECSpotETF2024}.

Die Literatur ordnet Bitcoin überwiegend als spekulatives Finanzmarktinstrument ein \citep{Baur2018}, wobei neuere Arbeiten auf eine zunehmende Integration in das globale Finanzsystem hinweisen \citep{Corbet2019}. Diese Einschätzung steht im Einklang mit früheren Analysen, die Bitcoin weniger als klassische Währung denn als Anlageobjekt betrachten \citep{Yermack2015}. Da alternative Kryptowährungen wie Ethereum ebenfalls ein erhebliches Marktvolumen aufweisen und teilweise andere technische Eigenschaften besitzen, wird Ethereum im Rahmen der empirischen Analyse ergänzend als Robustheitstest berücksichtigt, um die Spezifität der Ergebnisse für Bitcoin besser einordnen zu können.

Obwohl Kryptowährungen insgesamt seit 2015 erheblich an Bedeutung gewonnen haben, fokussiert sich diese Arbeit ausschließlich auf Bitcoin. Diese Abgrenzung ist dadurch begründet, dass Bitcoin über den betrachteten Zeitraum hinweg die mit Abstand höchste Marktkapitalisierung, Liquidität und institutionelle Akzeptanz aufweist und als Referenzwert für den gesamten Kryptomarkt fungiert. Zudem bildet Bitcoin den Schwerpunkt bestehender empirischer Studien, wodurch eine bessere Vergleichbarkeit der Ergebnisse gewährleistet ist \citep{Baur2018,Corbet2018}.

\section{Ausgewählte makroökonomische Variablen}
Makroökonomische Rahmenbedinungen spielen eine zentrale Rolle für die Preisbildung an Finanzmärkten. Vor diesem Hintergrund wird untersucht wie sich diese Faktoren auf die Preisentwicklung von Bitcoin und zum Vergleich Ethereum auswirken, da diese wie bereits erwähnt überwiegend als Anlageobjekte betrachtet werden. Im Folgenden werden die ausgewählten makroökonomischen Variablen erläutert, die in der empirischen Analyse berücksichtigt werden. Die empirische Analyse in dieser Seminararbeit konzentriert sich auf US-amerikanische Indikatoren, da der US-Markt den größten Anteil am globalen Kryptowährungsmarkt aufweist. \citep{Corbet2019}.

\subsection{Inflation}
Die Inflationsrate ist die Größe, die die Kaufkraft von Geld misst und Investoren dazu veranlassen kann, alternative Anlageobjekte zu suchen, um ihr Vermögen zu schützen. Da Bitcoin eine begrenzte Gesamtmenge von 21 Millionen Coins aufweist, kann er als Inflationsschutz dienen.

\subsection{Zinsen}
Das Zinsniveau ist ein Indikator der Geldpolitik und beeinflusst die Attraktivität verschiedener Anlageklassen. Niedrige Zinssätze reduzieren die Opportunitätskosten risikoreicher Anlagen und können daher die Nachfrage nach Kryptowährungen erhöhen, während steigende Zinssätze Kapital in traditionelle Anlageformen lenken können. In der Literatur wird Bitcoin in diesem Zusammenhang häufig als sogenanntes Risk-On-Asset eingeordnet, dessen Preisentwicklung sensibel auf geldpolitische Veränderungen reagiert \citep{Corbet2019}.

\subsection{Geldmenge}
Die Entwicklung der Geldmenge zeigt die Liquiditätsbedingungen innerhalb einer Volkswirtschaft und steht in engem Zusammenhang mit geldpolitischen Maßnahmen. Eine expansive Geldpolitik, die mit einer Ausweitung der Geldmenge einhergeht, kann zu einer erhöhten Nachfrage nach alternativen Anlageformen führen, insbesondere wenn Inflations- oder Abwertungsängste bestehen. Kryptowährungen werden in diesem Kontext teilweise als alternatives Wertaufbewahrungsmittel betrachtet, weshalb die Geldmenge als potenzieller Einflussfaktor berücksichtigt wird.

\subsection{US-Dollar}
Der US-Dollar nimmt eine zentrale Rolle im internationalen Finanzsystem ein und dient als wichtigste Reserve- und Handelswährung. Da Bitcoin und Ethereum überwiegend in US-Dollar notiert und gehandelt werden, können Wechselkursentwicklungen einen Einfluss auf deren Preisbildung haben. Eine Aufwertung des US-Dollars kann die relative Attraktivität von Kryptowährungen für internationale Investoren verringern, während eine Abschwächung des Dollars die Nachfrage begünstigen könnte.

\subsection{S\&P 500}
Der S\&P~500 ist ein Aktienindex, der die Wertentwicklung der 500 größten börsennotierten US-amerikanischen Unternehmen abbildet und daher als zentraler Indikator für die Entwicklung des US-amerikanischen Aktienmarktes gilt. Die Analyse des Zusammenhangs zwischen dem S\&P~500 und Kryptowährungen ermöglicht es, zu untersuchen, ob Kryptowährungen eher als Risikoanlagen in das allgemeine Marktgeschehen eingebunden sind oder in bestimmten Marktphasen Absicherungseigenschaften gegenüber herkömmlichen Finanzmärkten aufweisen.
