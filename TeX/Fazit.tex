\chapter{Fazit}

Ziel dieser Seminararbeit war es, den Zusammenhang zwischen ausgewählten makroökonomischen Variablen und der Preisentwicklung von Kryptowährungen zu untersuchen. Im Fokus stand dabei Bitcoin, der aufgrund seiner hohen Marktkapitalisierung, Liquidität und institutionellen Akzeptanz als stellvertretend für den Kryptomarkt betrachtet wurde. Ethereum wurde ergänzend als Robustheitstest herangezogen.

Die empirische Analyse zeigt, dass insbesondere die Renditen des S\&P~500 einen signifikanten und positiven Einfluss auf die Renditen von Bitcoin und Ethereum haben. Dieses Ergebnis deutet darauf hin, dass Kryptowährungen im betrachteten Zeitraum überwiegend als Risikoanlagen einzuordnen waren und sich ähnlich wie traditionelle Aktienmärkte entwickelten. Damit werden Kryptowährungen weniger als Absicherungsinstrumente, sondern vielmehr als spekulative Finanzmarktinstrumente eingeordnet, was im Einklang mit den Befunden früherer Studien steht \cite{Baur2018,Corbet2018,Corbet2019}.

Für andere makroökonomische Variablen wie Inflation, Geldmengenwachstum, US-Dollar-Index und Zinssätze konnten hingegen keine stabilen oder signifikanten Zusammenhänge mit den Renditen von Bitcoin festgestellt werden. Insbesondere liefert die Analyse keine eindeutigen Beweise dafür, dass Bitcoin als verlässlicher Inflationsschutz fungiert. Diese Ergebnisse stehen im Einklang mit der bestehenden Literatur, die Bitcoin keine konsistenten geldähnlichen Eigenschaften zuschreibt \cite{Baur2018,Yermack2015}.

Die Robustheitsanalyse mit Ethereum bestätigt die zentralen Ergebnisse der Bitcoin-Analyse. Sowohl Richtung als auch Signifikanz der geschätzten Koeffizienten bleiben konsistent, was die Stabilität der empirischen Analyse unterstreicht.

\section{Limitationen}

Trotz der plausiblen Ergebnisse unterliegt die vorliegende Analyse mehreren Einschränkungen. Es werden monatliche Zeitreihendaten verwendet, wodurch kurzfristige Marktreaktionen und Entwicklungen innerhalb eines Monats nicht abgebildet werden können. Eine Auswertung mit höher aufgelösten Daten könnte daher zusätzliche Erkenntnisse liefern.

Zudem beschränkt sich die Auswahl der erklärenden Variablen auf zentrale makroökonomische Indikatoren. Weitere Einflussfaktoren wie regulatorische Entwicklungen, technologische Ereignisse im Kryptomarkt oder marktinterne Kennzahlen wurden nicht berücksichtigt, könnten jedoch eine wichtige Rolle für die Preisentwicklung von Kryptowährungen spielen.

Zuletzt wird in der Regressionsanalyse ein lineares Modell verwendet. Nichtlineare Zusammenhänge oder zeitvariierende Effekte bleiben damit unberücksichtigt. Darüber hinaus wird keine explizite Kausalität untersucht, sodass die Ergebnisse primär als statistische Zusammenhänge zu interpretieren sind.

\section{Ausblick}
Zukünftige Forschung könnte an mehreren Stellen ansetzen. Eine mögliche Erweiterung besteht in der Verwendung höherfrequenter Daten, um kurzfristige Reaktionen von Kryptowährungen auf makroökonomische Ankündigungen detaillierter zu analysieren. Darüber hinaus könnten alternative ökonometrische Methoden wie Vektorautoregressionsmodelle (VAR) oder Regimewechselmodelle eingesetzt werden, um dynamische Wechselwirkungen besser abzubilden.

Außerdem wäre eine Ausweitung der Analyse auf weitere Kryptowährungen oder auf unterschiedliche Marktphasen sinnvoll, um strukturelle Unterschiede innerhalb des Kryptomarktes zu untersuchen. Schließlich könnte die zunehmende institutionelle Integration von Kryptowährungen, etwa durch börsengehandelte Produkte, künftig neue Zusammenhänge zwischen traditionellen Finanzmärkten und digitalen Vermögenswerten hervorbringen, die eine erneute empirische Untersuchung erforderlich machen.
