\chapter{Fazit}
Ziel dieser Seminararbeit war es, den Zusammenhang zwischen ausgewählten makroökonomischen Variablen und der Preisentwicklung von Kryptowährungen empirisch zu untersuchen. Im Fokus stand dabei Bitcoin, der aufgrund seiner hohen Marktkapitalisierung, Liquidität und zunehmenden institutionellen Akzeptanz als repräsentativ für den Kryptomarkt betrachtet wurde. Ethereum wurde ergänzend im Rahmen einer Robustheitsanalyse herangezogen.
Die empirische Analyse zeigt, dass insbesondere die Renditen des S\&P~500 einen signifikanten und positiven Einfluss auf die Renditen von Bitcoin und Ethereum haben. Dieses Ergebnis deutet darauf hin, dass Kryptowährungen im betrachteten Zeitraum überwiegend als Risikoanlagen einzuordnen waren und sich in ihrer kurzfristigen Renditeentwicklung ähnlich wie traditionelle Aktienmärkte verhielten. Damit werden Kryptowährungen weniger als Absicherungsinstrumente, sondern vielmehr als spekulative Finanzmarktinstrumente interpretiert, was im Einklang mit den Befunden früherer Studien steht \cite{Baur2018,Corbet2018,Corbet2019}.
Für andere makroökonomische Variablen wie Inflation, Geldmengenwachstum, den US-Dollar-Index sowie kurzfristige Zinsänderungen konnten hingegen keine stabilen oder statistisch signifikanten Zusammenhänge mit den Renditen von Bitcoin festgestellt werden. Insbesondere liefert die Analyse keine belastbare Evidenz dafür, dass Bitcoin im untersuchten Zeitraum als verlässlicher Inflationsschutz fungiert. Diese Ergebnisse stehen im Einklang mit der bestehenden Literatur, die Bitcoin keine konsistenten geldähnlichen Eigenschaften zuschreibt und seine Rolle als Wertaufbewahrungsmittel kritisch bewertet \cite{Baur2018,Yermack2015}.
Die Robustheitsanalyse mit Ethereum bestätigt die zentralen Ergebnisse der Bitcoin-Analyse. Sowohl die Richtung als auch die statistische Signifikanz der geschätzten Koeffizienten bleiben weitgehend konsistent, was auf eine gewisse Stabilität der empirischen Ergebnisse über verschiedene Kryptowährungen hinweg hindeutet.

\section{Limitationen}
Trotz der plausiblen Ergebnisse unterliegt die vorliegende Analyse mehreren Einschränkungen. Es werden monatliche Zeitreihendaten verwendet, wodurch kurzfristige Marktreaktionen auf makroökonomische Ankündigungen oder geldpolitische Entscheidungen innerhalb eines Monats nicht abgebildet werden können. Eine Auswertung mit höherfrequenten Daten könnte daher zusätzliche Erkenntnisse liefern.
Zudem beschränkt sich die Auswahl der erklärenden Variablen auf zentrale makroökonomische Indikatoren. Weitere potenziell relevante Einflussfaktoren wie regulatorische Entwicklungen, technologische Ereignisse im Kryptomarkt oder marktinterne Kennzahlen wurden nicht berücksichtigt, könnten jedoch einen wesentlichen Einfluss auf die Preisentwicklung von Kryptowährungen ausüben.
Zuletzt basiert die Regressionsanalyse auf einem linearen Modell. Mögliche nichtlineare Zusammenhänge, strukturelle Brüche oder zeitvariierende Effekte bleiben damit unberücksichtigt. Darüber hinaus wird keine explizite Kausalität untersucht, sodass die Ergebnisse primär als statistische Zusammenhänge und nicht als kausale Effekte zu interpretieren sind.

\section{Ausblick}
Zukünftige Forschung könnte an mehreren Stellen ansetzen. Eine mögliche Erweiterung besteht in der Verwendung höherfrequenter Daten, um kurzfristige Reaktionen von Kryptowährungen auf makroökonomische Ankündigungen oder geldpolitische Entscheidungen detaillierter zu analysieren. Darüber hinaus könnten alternative ökonometrische Methoden wie Vektorautoregressionsmodelle (VAR) oder Regimewechselmodelle eingesetzt werden, um dynamische Wechselwirkungen und potenzielle Strukturbrüche besser abzubilden.
Außerdem wäre eine Ausweitung der Analyse auf weitere Kryptowährungen oder unterschiedliche Marktphasen sinnvoll, um heterogene Effekte innerhalb des Kryptomarktes systematisch zu untersuchen. Schließlich könnte die zunehmende institutionelle Integration von Kryptowährungen, etwa durch börsengehandelte Produkte oder regulatorische Anpassungen, künftig neue Zusammenhänge zwischen traditionellen Finanzmärkten und digitalen Vermögenswerten hervorbringen, die eine erneute empirische Untersuchung erforderlich machen.
